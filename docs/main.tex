%%%%%%%%%%%%%%%%%%%%%%%%%%%%%%%%%%%%%%%%%
% Masters/Doctoral Thesis 
% LaTeX Template
% Version 2.5 (27/8/17)
%
% This template was downloaded from:
% http://www.LaTeXTemplates.com
%
% Version 2.x major modifications by:
% Vel (vel@latextemplates.com)
%
% This template is based on a template by:
% Steve Gunn (http://users.ecs.soton.ac.uk/srg/softwaretools/document/templates/)
% Sunil Patel (http://www.sunilpatel.co.uk/thesis-template/)
%
% Template license:
% CC BY-NC-SA 3.0 (http://creativecommons.org/licenses/by-nc-sa/3.0/)
%
%%%%%%%%%%%%%%%%%%%%%%%%%%%%%%%%%%%%%%%%%

%----------------------------------------------------------------------------------------
%	PACKAGES AND OTHER DOCUMENT CONFIGURATIONS
%----------------------------------------------------------------------------------------

\PassOptionsToPackage{greek,main=english}{babel}

\documentclass[
11pt, % The default document font size, options: 10pt, 11pt, 12pt
%oneside, % Two side (alternating margins) for binding by default, uncomment to switch to one side
english, % ngerman for German
onehalfspacing, % Single line spacing, alternatives: onehalfspacing or doublespacing
%draft, % Uncomment to enable draft mode (no pictures, no links, overfull hboxes indicated)
%nolistspacing, % If the document is onehalfspacing or doublespacing, uncomment this to set spacing in lists to single
%liststotoc, % Uncomment to add the list of figures/tables/etc to the table of contents
%toctotoc, % Uncomment to add the main table of contents to the table of contents
parskip, % Uncomment to add space between paragraphs
%nohyperref, % Uncomment to not load the hyperref package
headsepline, % Uncomment to get a line under the header
%chapterinoneline, % Uncomment to place the chapter title next to the number on one line
%consistentlayout, % Uncomment to change the layout of the declaration, abstract and acknowledgements pages to match the default layout
]{MastersDoctoralThesis} % The class file specifying the document structure

\let\thesisdegree=\degree
\undef\degree

\usepackage[T1]{fontenc} % Output font encoding for international characters
\usepackage[utf8]{inputenc} % Required for inputting international characters

\usepackage{mathpazo} % Use the Palatino font by default
\usepackage{comment}
\usepackage{lscape}
\usepackage{minted}
\usemintedstyle{friendly}
\usepackage{xcolor}
\definecolor{light-gray}{gray}{0.95}
\usepackage{amsmath}
\usepackage{gensymb}
\setlength\parindent{0pt}
\usepackage{microtype}
\usepackage{hyperref}
\usepackage[nameinlink,capitalise]{cleveref}


\usepackage[backend=bibtex,style=numeric,natbib=true]{biblatex} % Use the bibtex backend with the authoryear citation style (which resembles APA)

\addbibresource{example.bib} % The filename of the bibliography

\usepackage[autostyle=true]{csquotes} % Required to generate language-dependent quotes in the bibliography

%----------------------------------------------------------------------------------------
%	MARGIN SETTINGS
%----------------------------------------------------------------------------------------

\geometry{
	paper=a4paper, % Change to letterpaper for US letter
	inner=2.5cm, % Inner margin
	outer=3.8cm, % Outer margin
	bindingoffset=.5cm, % Binding offset
	top=1.5cm, % Top margin
	bottom=1.5cm, % Bottom margin
	%showframe, % Uncomment to show how the type block is set on the page
}

%----------------------------------------------------------------------------------------
%	THESIS INFORMATION
%----------------------------------------------------------------------------------------

\thesistitle{Image based orbit determination of the Didymos-Dimorphos binary asteroid system using the HERA spacecraft} % Your thesis title, this is used in the title and abstract, print it elsewhere with \ttitle
\supervisor{Prof. Kleomenis \textsc{Tsiganis}} % Your supervisor's name, this is used in the title page, print it elsewhere with \supname
\examiner{} % Your examiner's name, this is not currently used anywhere in the template, print it elsewhere with \examname
\thesisdegree{Computational Physics} % Your degree name, this is used in the title page and abstract, print it elsewhere with \degreename
\author{Anastasios-Faidon \textsc{Retselis}} % Your name, this is used in the title page and abstract, print it elsewhere with \authorname
\addresses{} % Your address, this is not currently used anywhere in the template, print it elsewhere with \addressname

\subject{Physics} % Your subject area, this is not currently used anywhere in the template, print it elsewhere with \subjectname
\keywords{} % Keywords for your thesis, this is not currently used anywhere in the template, print it elsewhere with \keywordnames
\university{\href{http://www.auth.gr}{Aristotle University of Thessaloniki}} % Your university's name and URL, this is used in the title page and abstract, print it elsewhere with \univname
\department{\href{https://www.physics.auth.gr/}{Physics Department}} % Your department's name and URL, this is used in the title page and abstract, print it elsewhere with \deptname
\group{\href{http://acubesat.asat.gr}{AcubeSAT Project}} % Your research group's name and URL, this is used in the title page, print it elsewhere with \groupname
\faculty{\href{http://www.sci.auth.gr}{Faculty of Sciences}} % Your faculty's name and URL, this is used in the title page and abstract, print it elsewhere with \facname

\AtBeginDocument{
\hypersetup{pdftitle=\ttitle} % Set the PDF's title to your title
\hypersetup{pdfauthor=\authorname} % Set the PDF's author to your name
\hypersetup{pdfkeywords=\keywordnames} % Set the PDF's keywords to your keywords
\hypersetup{bookmarksnumbered=true} % Add numbering to PDF bookmarks
}

\begin{document}
\hypersetup{urlcolor=blue}
\hypersetup{linkcolor=blue}
\hypersetup{citecolor=cyan}
\frontmatter % Use roman page numbering style (i, ii, iii, iv...) for the pre-content pages


\pagestyle{plain} % Default to the plain heading style until the thesis style is called for the body content

%----------------------------------------------------------------------------------------
%	TITLE PAGE
%----------------------------------------------------------------------------------------

\begin{titlepage}
\begin{center}

\begin{figure}[H]
\centering
\includegraphics[scale=0.4]{Figures/LogoAUTH72ppi.png}
%\label{fig:auth_logo}
\end{figure}

\vspace*{.04\textheight}
{\scshape\LARGE \univname\par}\vspace{1cm} % University name
\textsc{\Large Master Thesis}\\[0.5cm] % Thesis type

\HRule \\[0.4cm] % Horizontal line
{\huge \bfseries \ttitle\par}\vspace{0.4cm} % Thesis title
\HRule \\[1cm] % Horizontal line
 
\begin{minipage}[t]{0.4\textwidth}
\begin{flushleft} \large
\emph{Author:}\\
\href{https://gitlab.com/retse}{Anastasios-Faidon \textsc{Retselis}}
\end{flushleft}
\end{minipage}
\begin{minipage}[t]{0.4\textwidth}
\begin{flushright} \large
\emph{Supervisor:} \\
\href{http://users.auth.gr/tsiganis/}{Prof. Kleomenis \textsc{Tsiganis}}
\end{flushright}
\end{minipage}\\[1cm]
 
\vfill

\large \textit{A thesis submitted in fulfillment of the requirements\\ for the Master of Science (MSc) degree in \degreename}\\[0.3cm] % University requirement text
\textit{in the}\\[0.4cm]
\facname\\\deptname\\[1cm] % Research group name and department name
 
\vfill

{\large \today}\\[4cm] % Date
%\includegraphics{Logo} % University/department logo - uncomment to place it
 
\vfill
\end{center}
\end{titlepage}

%----------------------------------------------------------------------------------------
%	DECLARATION PAGE
%----------------------------------------------------------------------------------------

%\begin{declaration}
%\addchaptertocentry{\authorshipname} % Add the declaration to the table of contents
%\noindent I, \authorname, declare that this thesis titled, \enquote{\ttitle} and the work presented in it are my own. I confirm that:

%\begin{itemize} 
%\item This work was done wholly or mainly while in candidature for a research degree at this University.
%\item Where any part of this thesis has previously been submitted for a degree or any other qualification at this University or any other institution, this has been clearly stated.
%\item Where I have consulted the published work of others, this is always clearly attributed.
%\item Where I have quoted from the work of others, the source is always given. With the exception of such quotations, this thesis is entirely my own work.
%\item I have acknowledged all main sources of help.
%\item Where the thesis is based on work done by myself jointly with others, I have made clear exactly what was done by others and what I have contributed myself.\\
%\end{itemize}
 
%\noindent Signed:\\
%\rule[0.5em]{25em}{0.5pt} % This prints a line for the signature
 
%\noindent Date:\\
%\rule[0.5em]{25em}{0.5pt} % This prints a line to write the date
%\end{declaration}

%cleardoublepage
\cleardoublepage

%----------------------------------------------------------------------------------------
%	QUOTATION PAGE
%----------------------------------------------------------------------------------------

%\vspace*{0.2\textheight}

%\noindent\enquote{\itshape Thanks to my solid academic training, today I can write hundreds of words on virtually any topic without possessing a shred of information, which is how I got a good job in journalism.}\bigbreak

%\hfill Dave Barry

%----------------------------------------------------------------------------------------
%	ABSTRACT PAGE
%----------------------------------------------------------------------------------------

\begin{abstract}{\byname{} Asimakis \textsc{Anthopoulos} and Anastasios-Faidon \textsc{Retselis}}
\addchaptertocentry{\abstractname} % Add the abstract to the table of contents
The number of small- and nano-satellites being launched into orbit is expected to continue to grow in the coming years. In order to avoid the creation of space debris, spacecraft developers have to take action in order to reduce the severity of space debris. A widely used guideline is to ensure that no space system is left in the Low Earth Orbit environment for more than 25 years after the end of mission. Given this guideline, this thesis investigates 1U, 2U and 3U CubeSats in Low Earth Orbit using the General Mission Analysis Tool (GMAT) in order to determine the maximum allowable value of the semi-major axis so that the 25 year limit is met. This analysis results in orbital decay diagrams which can be used by CubeSat developers in order to evaluate different orbits early in the design phase.

These findings are then applied to the AcubeSAT mission in order to perform an orbital analysis tailored to the mission. AcubeSAT is a 3U CubeSat that is currently being designed by students in the Aristotle University of Thessaloniki with the support of the Education Office of the European Space Agency, under the educational Fly Your Satellite! Programme. The objective of AcubeSAT's mission is to probe the expression of eukaryotic genes in the environment of Low Earth Orbit. AcubeSAT's design calls for a very specific orientation to be achieved in order to downlink all images to the ground segment. This orientation is evaluated using the theory of spin-orbit coupling to determine if this orientation can be maintained in the event of a failure in the Attitude Determination \& Control Subsystem of the spacecraft. Based on the findings, an alternative design solution is also proposed.

\end{abstract}

{
    \makeatletter
    \def\facname{\foreignlanguage{greek}{\href{www.sci.auth.gr}{Σχολή Θετικών Επιστημών}}}
    \def\deptname{\foreignlanguage{greek}{\href{www.physics.auth.gr}{Τμήμα Φυσικής}}}
    \def\univname{\foreignlanguage{greek}{\href{www.auth.gr}{Αριστοτέλειο Πανεπιστήμιο Θεσσαλονίκης}}}
    \def\abstractname{\foreignlanguage{greek}{Περίληψη}}
    \def\@title{\foreignlanguage{greek}{Μελέτη τροχιάς και το πρόβλημα σύζευξης σπιν-τροχιάς για την αποστολή} AcubeSAT}
    \fontfamily{artemisia}\selectfont
    
    \begin{abstract}{\foreignlanguage{greek}{Ασημάκης \textsc{Ανθοπουλος} και Αναστάσιος-Φαίδων \textsc{Ρετσελης}}}
    \addchaptertocentry{{\fontfamily{artemisia}\selectfont\abstractname}} % Add the abstract to the table of contents
    \foreignlanguage{greek}{Ο αριθμός μικρο- και νανοδορυφόρων που πρόκειται να εκτοξευθούν σε τροχιά αναμένεται να αυξηθεί στα επόμενα χρόνια. Για να αποφευχθεί η δημιουργία διαστημικών σκουπιδιών, οι κατασκευαστές των διαστημικών σκαφών θα πρέπει να προβούν στις κατάλληλες πράξεις για να μειώσουν τον κίνδυνο των διαστημικών σκουπιδιών. Για το σκοπό αυτό χρησιμοποιείται ευρύτατα η οδηγία που αναφέρει πως κανένα διαστημικό σύστημα δεν πρέπει να μείνει στο περιβάλλον χαμηλής γήινης τροχιάς} (LEO) \foreignlanguage{greek}{για χρονικό διάστημα μεγαλύτερο των 25 ετών μετά το πέρας της αποστολής. Η εργασία αυτή ερευνά την συγκεκριμένη οδηγία για την περίπτωση των} 1U, 2U \foreignlanguage{greek}{και} 3U CubeSats \foreignlanguage{greek}{στο περιβάλλον της χαμηλής γήινης τροχιάς χρησιμοποιώντας το εργαλείο} General Mission Analysis Tool (GMAT)\foreignlanguage{greek}{, ούτως ώστε να βρεθούν τα ανώτατα όρια του μεγάλου ημιάξονα για να βρίσκεται η τροχιά κάτω από το όριο των 25 ετών}. \foreignlanguage{greek}{Η ανάλυση έχει ως αποτέλεσμα διαγράμματα που δείχνουν το χρόνο ζωής σε τροχιά συναρτήσει του αρχικού υψομέτρου, τα οποία μπορούν να χρησιμοποιηθούν από κατασκευαστές} CubeSat \foreignlanguage{greek}{σε αρχικά στάδια σχεδιασμού για να μελετηθούν διαφορετικές τροχιές.}
    
    \foreignlanguage{greek}{Τα αποτελέσματα αυτά χρησιμοποιούνται για την μελέτη της τροχιάς της αποστολής} AcubeSAT. \foreignlanguage{greek}{Το} AcubeSAT \foreignlanguage{greek}{είναι ένα} 3U CubeSAT \foreignlanguage{greek}{το οποίο σχεδιάζεται από φοιτητές του Αριστοτέλειου Πανεπιστημίου Θεσσαλονίκης με την υποστήριξη του εκπαιδευτικού γραφείου του Ευρωπαϊκού Οργανισμού Διαστήματος, υπό την αιγίδα του προγράμματος} Fly Your Satellite!. \foreignlanguage{greek}{H αποστολή} AcubeSAT \foreignlanguage{greek}{έχει ως στόχο να μελετήσει τη γονιδιακή έκφραση ευκαρυωτικών κυττάρων στο περιβάλλον της χαμηλής γήινης τροχιάς. Ο σχεδιασμός του} AcubeSAT \foreignlanguage{greek}{χρησιμοποιεί έναν συγκεκριμένο προσανατολισμό για να μπορέσει να στείλει τις εικόνες του πειράματος πίσω στη Γη. Ο προσανατολισμός αυτός διερευνάται με τη χρήση του προβλήματος σπιν-τροχιάς για να διευκρινιστεί εαν μπορεί να διατηρηθεί σε περίπτωση αποτυχίας του συστήματος ελέγχου προσανατολισμού του δορυφόρου. Με βάση τα αποτελέσματα, προτείνεται και ένας διαφορετικός σχεδιασμός για την αποστολή.} 
    \end{abstract}
    \makeatother
}
%----------------------------------------------------------------------------------------
%	ACKNOWLEDGEMENTS
%----------------------------------------------------------------------------------------

\begin{acknowledgements}
\addchaptertocentry{\acknowledgementname} % Add the acknowledgements to the table of contents
Firstly, we would like to thank our supervisors Prof. Georgios Voygiatzis and Prof. Kleomenis Tsiganis for their continuous support and guidance throughout the research phase of this thesis. Furthermore, we would like to thank all of our colleagues working at the AcubeSAT project for enabling us to write this thesis by working hard to design all aspects of the spacecraft for the past 4 years. Special thanks to Konstantinos Kanavouras for his help in solving some \LaTeX\  issues in this thesis. Finally, we would like to thank our loved ones for supporting us throughout the years. We would have not been able to make it without you.

\vspace*{0.2\textheight}
\noindent\enquote{\itshape If it's not fun, why bother...}\bigbreak

\hfill - Reginald "Reggie" Fils-Aimé

\vspace*{0.1\textheight}

\noindent\enquote{\itshape Everything is fun, but when you get to the shit, it is fucking shit.}\bigbreak

\hfill - An anonymous AcubeSAT member during a high pressure period.

\end{acknowledgements}

%----------------------------------------------------------------------------------------
%	LIST OF CONTENTS/FIGURES/TABLES PAGES
%----------------------------------------------------------------------------------------

\tableofcontents % Prints the main table of contents

\listoffigures % Prints the list of figures

\listoftables % Prints the list of tables

%----------------------------------------------------------------------------------------
%	ABBREVIATIONS
%----------------------------------------------------------------------------------------

%\begin{abbreviations}{ll} % Include a list of abbreviations (a table of two columns)

%\textbf{LAH} & \textbf{L}ist \textbf{A}bbreviations \textbf{H}ere\\
%\textbf{WSF} & \textbf{W}hat (it) \textbf{S}tands \textbf{F}or\\
%
%\end{abbreviations}

%----------------------------------------------------------------------------------------
%	PHYSICAL CONSTANTS/OTHER DEFINITIONS
%----------------------------------------------------------------------------------------

%\begin{constants}{lr@{${}={}$}l} % The list of physical constants is a three column table

% The \SI{}{} command is provided by the siunitx package, see its documentation for instructions on how to use it

%Speed of Light & $c_{0}$ & \SI{2.99792458e8}{\meter\per\second} (exact)\\
%Constant Name & $Symbol$ & $Constant Value$ with units\\

%\end{constants}

%----------------------------------------------------------------------------------------
%	SYMBOLS
%----------------------------------------------------------------------------------------

%\begin{symbols}{lll} % Include a list of Symbols (a three column table)

%$a$ & distance & \si{\meter} \\
%$P$ & power & \si{\watt} (\si{\joule\per\second}) \\
%Symbol & Name & Unit \\

%\addlinespace % Gap to separate the Roman symbols from the Greek

%$\omega$ & angular frequency & \si{\radian} \\

%\end{symbols}

%----------------------------------------------------------------------------------------
%	DEDICATION
%----------------------------------------------------------------------------------------

%\dedicatory{For/Dedicated to/To my\ldots} 

%----------------------------------------------------------------------------------------
%	THESIS CONTENT - CHAPTERS
%----------------------------------------------------------------------------------------

\mainmatter % Begin numeric (1,2,3...) page numbering

\pagestyle{thesis} % Return the page headers back to the "thesis" style

% Include the chapters of the thesis as separate files from the Chapters folder
% Uncomment the lines as you write the chapters

%\chapter{Introduction}

\section{Planetary defense}
\label{sec:planetary_defense}

The final law provided in Akin's Laws of Spacecraft Design states that \textit{"Space is a completely unforgiving environment. If you screw up the engineering, somebody dies."} \cite{akinslaws}. This law was written for engineers designing and building crewed space systems and it highlights the facts that space is an extremely hostile environment towards humans. However, the issue at hand is that space is also hostile to humans even when they are on their own planet. Of significant risk to planet Earth is a potential collision with a large asteroid or near-Earth object (NEO), which would cause extreme environmental phenomena massive tsunamis, multiple firestorms and potentially an impact winter caused by dust particles and debris rising to the stratosphere. All of this would ultimately lead to a mass extinction event, in which several species and life forms on our planet could not survive on the planet anymore. In fact, this has already happened once in the history of the Earth around 66 million years ago, when an object about 10 kilometers wide hit our planet in North America and triggered the \textit{Cretaceous-Paleogene} extinction event, which is the event that caused the extinction of most dinosaurs.

\subsection{Near-Earth Objects}
\label{ssec:neo}

Fortunately, unlike dinosaurs, humanity has created and developed resources dedicated to cataloging near-Earth objects, estimating the likelihood of Earth impacts and developing methods to avoid a potential collision. These resources are usually realized as a dedicated office within a larger government mandated space agency. Following the conventions used by the European Space Agency's Near-Earth Objects Coordination Centre (NEO), we can define \textbf{near-Earth objects} as asteroids or comets with a perihelion distance of \num{\leq1.3} \si{\astronomicalunit}, with comets having the additional requirement of having a period shorter than \num{200} years \cite{neo_definition}. Furthermore, we can define \textbf{Potentially Hazardous Asteroids (PHAs)} as asteroids that have a Earth Minimum Orbit Intersection Distance (MOID) of \num{0.05} \si{\astronomicalunit} or less, combined with an absolute magnitude $H$ of \num{22} or brighter \cite{neo_definition}. 


NASA's Jet Propulsion Laboratory has a dedicated Center for Near Earth Object Studies (CNEOS). CNEOS publishes rough statistics to monitor the progress of annual NEA discoveries, which can be seen in \Cref{fig:NEA_nasa}. By comparing JPL's data with data from IAU's Minor Planet Center, it can be determined that from \num{28500} objects discovered as of March 2022, \num{2271} of them are characterized as Potentially Hazardous Asteroids (PHAs) \cite{mpc_data}. 

\begin{figure}[h]
	\centering
	\includegraphics[width=\textwidth]{Figures/Chapter1/jpl_nea_data.pdf}
	\caption{Near-Earth Asteroids cataloged over the years. Today, about 8\% of them are classified as Potentially Hazardous Asteroids (data from \cite{nea_stats_jpl})}
	\label{fig:NEA_nasa}
\end{figure}

\subsection{Techniques for collision avoidance}
\label{ssec:collision_avoidance}


\section{The Asteroid Impact and Deflection Assessment (AIDA)}
%\include{Chapters/Chapter3}
%\chapter{Methodology \& Model design}
\label{chap:methodology}

\section{Programming Language}
\label{sec:language}

Given the computational nature of the orbit determination problem introduced in \Cref{chap:introduction}, a decision for the appropriate programming language in which the model will be developed has to be made. In today's academic environment, several scientists chose to develop their code in dynamic languages such as MATLAB or Python, which enhance productivity by giving a variety of tools to developers. However, these dynamic programming languages suffer during problems which can be classified as computationally intensive, leading to the use of languages such as C or Fortran when dealing computationally heavy problems \cite{Julia-2017}. 

The problem with the latter languages is that they do not offer the productivity provided by the dynamic languages, which have arguably made the development of scientific code fairly easier, leading to developers having to perform a trade-off for each problem to choose the appropriate language.  This problem is frequently described as the two language problem in literature. A solution which combines both the productivity and performance and thus solves the two language problem is Julia \cite{Julia-2017}. From \Cref{fig:julia_bench}, it can be concluded that code written in Julia will achieve similar benchmark times to C \cite{julia-benchmark}, while also providing the user with high level functions and options to write code in a more productive manner.

\begin{figure}[h]
	\centering
	\includegraphics[width=\textwidth]{Figures/Chapter2/julia_benchmarks.pdf}
	\caption{Normalized benchmark time of some problems against the C implementation for different programming languages \cite{julia-benchmark}}
	\label{fig:julia_bench}
\end{figure}

In the context of this thesis, it can be expected that the code developed will be computationally intensive, given the fact that numerical integrators used to perform the propagation of the asteroids for the fitting procedure (which will undoubtedly run several times until a solution is identified), leading to a need for performance. In addition, coordinate transformations, missing objects from the images generated and other pitfalls require a productive way to deal with these problems. Since the two language problem is being faced, the decision has been made early to use Julia as the programming language for the development of this thesis.




\section{Orbital Dynamics}

For the description of the orbit determination problem being faced from the perspective of astrodynamics, we can simplify the problem into two separate problems, namely:

\begin{enumerate}
	\item The orbit followed by the Hera spacecraft during the observation period.
	\item The behavior of the two asteroids Didymos \& Dimorphos.
\end{enumerate}

Once an adequate description for both of these sub-problems has been given, they can then combine them into a single dynamics model using the appropriate reference frames and transformations between them. After this combination is complete, the image generation procedure can be built and different types of errors can be added.  
    
%\include{Chapters/Chapter4} 
%\include{Chapters/Chapter5} 

%----------------------------------------------------------------------------------------
%	THESIS CONTENT - APPENDICES
%----------------------------------------------------------------------------------------

%\appendix % Cue to tell LaTeX that the following "chapters" are Appendices

% Include the appendices of the thesis as separate files from the Appendices folder
% Uncomment the lines as you write the Appendices

%\include{Appendices/AppendixGMAT}
%\include{Appendices/AppendixDecay}
%\include{Appendices/AppendixMissionAnalysis}
%\include{Appendices/AppendixMathematica}
%\include{Appendices/AppendixUnstabelSolution}
%\include{Appendices/AppendixStableSolution}

%----------------------------------------------------------------------------------------
%	BIBLIOGRAPHY
%----------------------------------------------------------------------------------------

\printbibliography[heading=bibintoc]

%----------------------------------------------------------------------------------------

\end{document}  
